\documentclass[12pt]{article}
\usepackage[utf8]{inputenc}
\usepackage[T1]{fontenc}
\usepackage{amsmath}
\usepackage{amsfonts}
\usepackage{amssymb}
\usepackage{amsthm}
\usepackage{fullpage}
\usepackage{lscape}
\usepackage{color}
\usepackage{algorithm}
\usepackage{algpseudocode}
\usepackage{url}
\usepackage{setspace}
\usepackage{enumitem}
%\usepackage[authoryear]{natbib}
%\usepackage[style=authoryear,natbib=true,isbn=false,doi=false]{biblatex}
\usepackage[american]{babel}
\usepackage[babel]{csquotes}
\usepackage[natbib,authordate,backend=biber,isbn=false,doi=false,noibid]{biblatex-chicago}
%\usepackage[authordate,backend=biber,isbn=false,doi=false,noibid]{biblatex-chicago}
\usepackage{flexisym}

% for natbib
\def\cite{\citep}

%% for biblatex
%\def\cite{\parencite}
%\def\citep{\parencite}
%\def\citet{\textcite}
%\def\citealt{\Cite}

\renewcommand*{\bibfont}{\footnotesize}

\allowdisplaybreaks

\newtheorem{theorem}{Theorem}
\newtheorem{lemma}{Lemma}
\newtheorem{proposition}{Proposition}
\newtheorem{corollary}{Corollary}
\newtheorem{remark}{Remark}
%\newtheorem{proof}{Proof}

\DeclareMathOperator{\prox}{\bf prox}
\DeclareMathOperator{\gra}{\bf gra}
\DeclareMathOperator{\zer}{\bf zer}
\DeclareMathOperator{\Fix}{\bf Fix}
\DeclareMathOperator{\dom}{\bf dom}
\DeclareMathOperator*{\argmin}{arg\,min}
\DeclareMathOperator{\diag}{diag}
\DeclareMathOperator{\Tr}{\bf Tr}
\DeclareMathOperator{\Expect}{\bf E}
\DeclareMathOperator{\Cov}{\bf Cov}
\DeclareMathOperator{\Ind}{\bf 1}
\newcommand{\ba}{\boldsymbol{a}}
\newcommand{\bb}{\boldsymbol{b}}
\newcommand{\bc}{\boldsymbol{c}}
\newcommand{\bd}{\boldsymbol{d}}
\newcommand{\be}{\boldsymbol{e}}
\newcommand{\bff}{\boldsymbol{f}}
\newcommand{\bg}{\boldsymbol{g}}
\newcommand{\bh}{\boldsymbol{h}}
\newcommand{\bi}{\boldsymbol{i}}
\newcommand{\bj}{\boldsymbol{j}}
\newcommand{\bk}{\boldsymbol{k}}
\newcommand{\bl}{\boldsymbol{l}}
\newcommand{\bm}{\boldsymbol{m}}
\newcommand{\bn}{\boldsymbol{n}}
\newcommand{\bo}{\boldsymbol{o}}
\newcommand{\bp}{\boldsymbol{p}}
\newcommand{\bq}{\boldsymbol{q}}
\newcommand{\br}{\boldsymbol{r}}
\newcommand{\bs}{\boldsymbol{s}}
\newcommand{\bt}{\boldsymbol{t}}
\newcommand{\bu}{\boldsymbol{u}}
\newcommand{\bv}{\boldsymbol{v}}
\newcommand{\bw}{\boldsymbol{w}}
\newcommand{\bx}{\boldsymbol{x}}
\newcommand{\by}{\boldsymbol{y}}
\newcommand{\bz}{\boldsymbol{z}}
\newcommand{\bA}{\boldsymbol{A}}
\newcommand{\bB}{\boldsymbol{B}}
\newcommand{\bC}{\boldsymbol{C}}
\newcommand{\bD}{\boldsymbol{D}}
\newcommand{\bE}{\boldsymbol{E}}
\newcommand{\bF}{\boldsymbol{F}}
\newcommand{\bG}{\boldsymbol{G}}
\newcommand{\bH}{\boldsymbol{H}}
\newcommand{\bI}{\boldsymbol{I}}
\newcommand{\bJ}{\boldsymbol{J}}
\newcommand{\bK}{\boldsymbol{K}}
\newcommand{\bL}{\boldsymbol{L}}
\newcommand{\bM}{\boldsymbol{M}}
\newcommand{\bN}{\boldsymbol{N}}
\newcommand{\bO}{\boldsymbol{O}}
\newcommand{\bP}{\boldsymbol{P}}
\newcommand{\bQ}{\boldsymbol{Q}}
\newcommand{\bR}{\boldsymbol{R}}
\newcommand{\bS}{\boldsymbol{S}}
\newcommand{\bT}{\boldsymbol{T}}
\newcommand{\bU}{\boldsymbol{U}}
\newcommand{\bV}{\boldsymbol{V}}
\newcommand{\bW}{\boldsymbol{W}}
\newcommand{\bX}{\boldsymbol{X}}
\newcommand{\bY}{\boldsymbol{Y}}
\newcommand{\bZ}{\boldsymbol{Z}}
\newcommand{\balpha}{\boldsymbol{\alpha}}
\newcommand{\bbeta}{\boldsymbol{\beta}}
\newcommand{\bgamma}{\boldsymbol{\gamma}}
\newcommand{\bdelta}{\boldsymbol{\delta}}
\newcommand{\bepsilon}{\boldsymbol{\epsilon}}
\newcommand{\blambda}{\boldsymbol{\lambda}}
\newcommand{\bmu}{\boldsymbol{\mu}}
\newcommand{\bnu}{\boldsymbol{\nu}}
\newcommand{\bphi}{\boldsymbol{\phi}}
\newcommand{\bpi}{\boldsymbol{\pi}}
\newcommand{\bsigma}{\boldsymbol{\sigma}}
\newcommand{\btheta}{\boldsymbol{\theta}}
\newcommand{\bomega}{\boldsymbol{\omega}}
\newcommand{\bxi}{\boldsymbol{\xi}}
\newcommand{\bGamma}{\boldsymbol{\rho}}
\newcommand{\bDelta}{\boldsymbol{\Delta}}
\newcommand{\bTheta}{\boldsymbol{\Theta}}
\newcommand{\bLambda}{\boldsymbol{\Lambda}}
\newcommand{\bXi}{\boldsymbol{\Xi}}
\newcommand{\bPi}{\boldsymbol{\Pi}}
\newcommand{\bOmega}{\boldsymbol{\Omega}}
\newcommand{\bUpsilon}{\boldsymbol{\Upsilon}}
\newcommand{\bPhi}{\boldsymbol{\Phi}}
\newcommand{\bPsi}{\boldsymbol{\Psi}}
\newcommand{\bSigma}{\boldsymbol{\Sigma}}

\title{Gauss Elimination and LU Decomposition Example}
\author{Joong-Ho Won, Computational Statistics, SNU}
%\date{Advanced Statistical Computing}
\date{}

\begin{document}

\maketitle

\onehalfspacing

\begin{center}
\begin{tabular}{l|l}
System of equations
&
Associated matrices
\\
\parbox{8cm}{%
	\begin{align*}
	2x_1 - 4x_2 + 2x_3 &= 6  \tag{1} \\
	4x_1 - 9x_2 + 7x_3 &= 20 \tag{2} \\
	2x_1 +  x_2 + 3x_3 &= 14 \tag{3}
	\end{align*}%
}
&
\parbox{8cm}{%
\[
	\underbrace{%
	\begin{bmatrix}
	2 & -4 & 2 \\
	4 & -9 & 7 \\
	2 &  1 & 3 	
	\end{bmatrix}%
	}_{A}
\]%
}
\\
\hline
(Step 1) &  \\
(2\textprime) = (2) - (1)$\times$ 2; (1\textprime)=(1); (3\textprime)=(3)
&
(2) = 2$\times$(1) + 1$\times$(2\textprime); (1)=(1\textprime); (3)=(3\textprime)
\\
\parbox{8cm}{%
	\begin{align*}
	2x_1 - 4x_2 + 2x_3 &= 6  \tag{1\textprime} \\
	     -  x_2 + 3x_3 &= 8 \tag{2\textprime} \\
	2x_1 +  x_2 + 3x_3 &= 14 \tag{3\textprime}
	\end{align*}%
}
&
\parbox{8cm}{%
\[
	\underbrace{%
	\begin{bmatrix}
	2 & -4 & 2 \\
	4 & -9 & 7 \\
	2 &  1 & 3 	
	\end{bmatrix}%
	}_{A}
	= 
	\underbrace{%
	\begin{bmatrix}
	1 & 0 & 0 \\
	2 & 1 & 0 \\
	0 & 0 & 1 	
	\end{bmatrix}
	}_{L_1}
	\underbrace{%
	\begin{bmatrix}
	2 & -4 & 2 \\
	0 & -1 & 3 \\
	2 &  1 & 3 	
	\end{bmatrix}%
	}_{A\textprime}
\]%
}
\\
\hline
(Step 2) &  \\
(3\textprime\textprime) = (3\textprime) - (1\textprime)$\times$ 1; (1\textprime\textprime)=(1\textprime); (2\textprime\textprime)=(2\textprime)
&
(3\textprime) = 1$\times$(1\textprime) + 1$\times$(3\textprime\textprime); (1\textprime)=(1\textprime\textprime); (2\textprime)=(2\textprime\textprime)
\\
\parbox{8cm}{%
	\begin{align*}
	2x_1 - 4x_2 + 2x_3 &= 6  \tag{1\textprime\textprime} \\
	     -  x_2 + 3x_3 &= 8 \tag{2\textprime\textprime} \\
	       5x_2 +  x_3 &= 8 \tag{3\textprime\textprime}
	\end{align*}%
}
&
\parbox{8cm}{%
\begin{align*}
	\underbrace{%
	\begin{bmatrix}
	2 & -4 & 2 \\
	0 & -1 & 3 \\
	2 &  1 & 3 	
	\end{bmatrix}%
	}_{A\textprime}
	&= 
	\underbrace{%
	\begin{bmatrix}
	1 & 0 & 0 \\
	0 & 1 & 0 \\
	1 & 0 & 1 	
	\end{bmatrix}
	}_{L_2}
	\underbrace{%
	\begin{bmatrix}
	2 & -4 & 2 \\
	0 & -1 & 3 \\
	0 &  5 & 1 	
	\end{bmatrix}%
	}_{A\textprime\textprime}
\end{align*}%
}
\\
\end{tabular}
\end{center}

\newpage

\begin{center}
\begin{tabular}{r|l}
(Step 3) &  \\
(3\textprime\textprime\textprime) = (3\textprime\textprime) - (2\textprime\textprime)$\times$ (-5); (1\textprime\textprime\textprime)=(1\textprime\textprime); (2\textprime\textprime\textprime)=(2\textprime\textprime)
&
(3\textprime\textprime) = -5$\times$(2\textprime\textprime) + $\times$(3\textprime\textprime\textprime); (1\textprime\textprime)=(1\textprime\textprime\textprime); (2\textprime\textprime)=(2\textprime\textprime\textprime)
\\
\parbox{8cm}{%
	\begin{align*}
	2x_1 - 4x_2 + 2x_3 &= 6  \tag{1\textprime\textprime\textprime} \\
	     -  x_2 + 3x_3 &= 8 \tag{2\textprime\textprime\textprime} \\
	             16x_3 &= 48 \tag{3\textprime\textprime\textprime}
	\end{align*}%
}
&
\parbox{8cm}{%
\begin{align*}
	\underbrace{%
	\begin{bmatrix}
	2 & -4 & 2 \\
	0 & -1 & 3 \\
	0 &  5 & 1 	
	\end{bmatrix}%
	}_{A\textprime\textprime}
	&= 
	\underbrace{%
	\begin{bmatrix}
	1 &  0 & 0 \\
	0 &  1 & 0 \\
	0 & -5 & 1 	
	\end{bmatrix}
	}_{L_3}
	\underbrace{%
	\begin{bmatrix}
	2 & -4 & 2 \\
	0 & -1 & 3 \\
	0 &  0 & 16	
	\end{bmatrix}%
	}_{A\textprime\textprime\textprime}
\end{align*}%
}
\\
\end{tabular}
\end{center}

Thus
\begin{align*}
	A &= L_1 A\textprime \\
	  &= L_1 L_2 A\textprime\textprime \\
	  &= L_1 L_2 L_3 A\textprime\textprime\textprime \\
	  &=
	\begin{bmatrix}
	1 & 0 & 0 \\
	2 & 1 & 0 \\
	0 & 0 & 1 	
	\end{bmatrix}
	\begin{bmatrix}
	1 & 0 & 0 \\
	0 & 1 & 0 \\
	1 & 0 & 1 	
	\end{bmatrix}
	\begin{bmatrix}
	1 &  0 & 0 \\
	0 &  1 & 0 \\
	0 & -5 & 1 	
	\end{bmatrix}
	\begin{bmatrix}
	2 & -4 & 2 \\
	0 & -1 & 3 \\
	0 &  0 & 16	
	\end{bmatrix}%
	  &=
	  \underbrace{%
	  \begin{bmatrix}
	  1 & 0 & 0 \\
	  2 & 1 & 0 \\
	  1 & -5 & 1
	  \end{bmatrix}%
	  }_{L}
	  \underbrace{%
	\begin{bmatrix}
	2 & -4 & 2 \\
	0 & -1 & 3 \\
	0 &  0 & 16	
	\end{bmatrix}%
	}_{U}
	.
\end{align*}
			
\end{document}

